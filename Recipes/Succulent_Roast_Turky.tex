\begin{Large}
    Roast Turky
\end{Large}

\begin{scriptsize}
\begin{multicols}{2}
[
\vspace{1em}
Yields: 4 Units
\vspace{-1.5em}
]

Dry Ingredients
\begin{itemize}
    \item 300 g (2 cup) vital wheat gluten
    \item 32 g (1/4 cup) all purpose flour
    \item 4 tsp onion powder
    \item 2 tsp garlic powder
\end{itemize}

Wet Ingredients
\begin{itemize}
    \item 280 g (10 oz) pressed extra firm block tofu
    \item 1 and 1/2 cup (355 ml) water
    \item 2 tbsp mild vegetable oil
    \item 50 g (2 tbsp) mellow white miso paste
    \item 1 tsp salt
    \item 1 tsp monosodium glutamate
    \item 1 tsp yeast extract
    \item 1/4 tsp sage
    \item 1/4 tsp thyme
    \item 1/4 tsp rosemary
    \item 1/4 tsp white pepper
    \item 1/8 tsp nutmeg
\end{itemize}

Broth Ingredients
\begin{itemize}
    \item 16 cup (3785 ml) water
    \item 4 large onions, peeled and quartered
    \item 4 ribs celery, chopped
    \item 2 carrots, unpeeled and chopped
    \item 1 handful parsley stems (leaves removed and saved for the pan-glaze garnish)
    \item 8 clove crushed garlic
    \item 2 tsp monosodium glutamate
    \item 2 tsp yeast extract
    \item 2 tbsp tamari, soy sauce or liquid aminos 
    \item 4 tsp salt
    \item 2 tsp sugar
    \item 2 tsp sage
    \item 2 tsp thyme
    \item 2 tsp rosemary
    \item 1 and 1/2 tsp peppercorns
    \item 2 bay leaves
\end{itemize}

Glaze Ingredients
\begin{itemize}
    \item 3 tbsp butter
    \item 1 tbsp tamari, soy sauce or liquid aminos
    \item 1/4 cup dry white wine or reserved broth
    \item 1 tsp sage
    \item 1 tsp thyme
    \item 1 tsp rosemary
    \item 1/4 tsp black pepper, to taste
\end{itemize}

Skin Ingredients
\begin{itemize}
    \item 1 pcs Vietnamese rice paper wrapper (round)
    \item 1/2 cup reserved vegetable broth
    \item 1-2 tbsp butter
\end{itemize}
\end{multicols}
\end{scriptsize}

\begin{footnotesize}

\noindent
\textbf{The roast}
\begin{enumerate}
    \item Preheat the oven to 350\degree F/180\degree C
    \item Combine the dry ingredients in a large mixing bowl
    \item Into a blender, crumble the tofu and add the remaining wet ingredients
    \item Process the contents until the tofu is completely liquefied. Stop the blender as necessary to scrape down the sides
    \item Pour the tofu mixture into the dry ingredients
    \item Fold the the wet into the dry until a ball of dough begins to form
    \item Knead the dough for 1 minute in a machine or 3 minutes by hand
    \item In a 18 by 24 inch sheet of foil, place the dough onto the foil and shape it into a cylinder
    \item Crimp the foil over the dough, then twist the ends like a candy wrapper and bend ends in half
    \item Wrap with a second sheet of foil and twist the ends tightly
    \item Place the package on a baking sheet and directly on the middle rack of the oven
    \item Bake for 1 hour and 30 minutes
    \item While the roast is pre-baking, prepare the simmering broth
    \item Add all of the broth ingredients to a large cooking pot and bring to a boil, once boiling reduce to a gentle simmer
    \item When the roast has finish, let cool for about 30 minutes
    \item Unwrap the roast and with a fork, pierce the roast 4 times on the top and 4 times on the bottom
    \item Remove any large solid ingredients from the simmer broth
    \item Carefully lower the roast into the broth and adjust the heat to keep a simmer
    \item Cook for 1 hour, turning the roast every 15 minutes
    \item Monitor the pot frequently and adjust the heat as necessary to maintain the simmer. The broth should be gently bubbling. Do not boil, but do not let the roast merely poach in hot liquid either, as a gentle simmer is necessary to penetrate the roast and finish the cooking process
    \item Remove the pot from the heat, cover and let cool until lukewarm
    \item Remove the roast, store in the refrigerate with 1/4 cup broth for a minimum of 8 hours or for up to 1 week before finishing
    \item To freeze, store the roast, with as little air as possible, without any broth for up to 1 month. Thaw completely before finishing
\end{enumerate}

\noindent
\textbf{Adding the skin and finishing the roast}
\begin{enumerate}
    \item Bring the roast to room temperature before finishing     
    \item Preheat the oven in the range of 350\degree F-425\degree F (180\degree C-218\degree C), depending on your oven
    \item Blot the roast with a paper towel
    \item Place the roast in a baking dish and coat lightly with flour
    \item In a shallow dish (pie pan), add 2-4 tbsp of broth
    \item Wet the rice paper in the broth until soft and pliable
    \item Drain the liquid from the dish and place the paper rough side up
    \item Brush butter or oil on the rough side that is facing up
    \item Place the buttered rough side directly on the roast
    \item Remove any air gaps and gently tuck the ends of the rice paper under the roast
    \item In a sauce pan melt the butter or margarine over medium heat
    \item Add the remaining glaze ingredients
    \item Whisk to combine and bring to a simmer to complete the glaze
    \item Brush or pour half of the glaze over the roast
    \item Cover in black pepper or your favorite spice mix/rub, if desired
    \item Bake for 20 minutes
    \item Pour or baste the remaining glaze on to the roast
    \item Bake for another 20 minutes or until crispy
    \item Transfer the roast to a serving platter
    \item Slice, serve, and enjoy it
\end{enumerate}
\end{footnotesize}

\vspace{2em}
