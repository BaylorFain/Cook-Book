\begin{Large}
    Lemon Curd
\end{Large}

\begin{scriptsize}
\begin{multicols}{2}
[
\vspace{1em}
Yields: 1 Unit
\vspace{-1.5em}
]

Ingredients
\begin{itemize}
    \item 1/2 cup (100 gr) granulated sugar
    \item 2 tablespoons (20 gr) cornstarch
    \item Pinch of salt
    \item 1 cup (240 ml) milk
    \item 1⁄2 cup (120 ml) fresh lemon juice (about 4-5 lemons)
    \item 1 tablespoon lemon zest
    \item 1 -2 drops yellow soft gel paste food color (optional)
    \item 2 tablespoons (30 gr) unsalted butter
%          If baking add 1/2 tsp agar
\end{itemize}
\end{multicols}
\end{scriptsize}

\begin{footnotesize}
\begin{enumerate}
    \item Zest the lemons
    \item Juice the lemons until you have about half a cup of lemon juice. Don't forget to remove out the seeds.
    \item Mix sugar, cornstarch, and salt in a small saucepan.
    \item Add milk, lemon juice, and lemon zest; mix to combine.
    \item Heat on low, while stirring constantly with a wire whisk until mixture thickens, just starts to bubble and coats back of a wooden spoon. NOTE: If the curd isn’t thickening, turn up the heat and constantly whisk.
    \item Remove pot from heat, then add the cubed butter and mix until melted. If desired, add 1 -2 drops of yellow gel food coloring to intensify the color. I added two drops of this yellow soft gel paste food color.
    \item Pour the lemon curd into a heatproof bowl, cover with plastic wrap pressed onto the top of the curd to avoid a skin from forming on top, and refrigerate until cold. The curd will continue to thicken as it cools. Once cool, the plastic wrap can be removed.

Do not use bottled lemon juice. Use fresh-squeezed lemon juice.
You can use salted butter instead of unsalted butter. Simply omit the pinch of salt called in the
recipe.Keep the heat low! And don't stop whisking until it's off heat.
You'll know the curd is ready when it noticeably thickens and coats the back of a spoon. You can
also use a thermometer to see when it reaches 160 degrees.
\end{enumerate}
\end{footnotesize}

\vspace{2em}
